\documentclass[12pt]{article}
\usepackage[a4paper]{geometry}
\usepackage[utf8]{inputenc}
\usepackage[russian]{babel}
\usepackage{amssymb}
\begin{document}
Теорема Кантора: не существует сюръективной функции, отображающей $\mathbb{A}$ в $\mathcal{P}(\mathbb{A})$. Следствие: для любого множества $\mathbb{A}$, $|\mathcal{P}(\mathbb{A})| > |\mathbb{A}|$. .

Базовое понятие --- множество, двуместный предикат ``принадлежность'' $a \in \mathbb{A}$
Функция из множества X в множество Y --- это правило, которое сопоставляеткаждому элементу множества X ровно 1 элемент из множества Y. Множество X называется доменом функции, Y --- кодоменом. Множество всех выходных значений называется range (диапазон). Множество всех выходных значений функции называется также ``образ функции'' (image).

Сюръективная функция (surjective function) --- это функция, такая что $\forall y \in codomain(f), \exists x \in domain(f), f(x) = y$.
\end{document}
